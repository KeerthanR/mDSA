\section{Experiments}
\label{sec:experiments}
A testsuite $\Pi(\Tt)$ for an EDT $\Tt$ is a set of executions such
that for each row $R$ of $\Tt$ there exists an execution
$\pi \in \Pi$ and $\pi$ triggers $R$. To find a testsuite, we need to solve the test generation problem for each row.
%The testsuite generation problem is to check whether there exists a testsuite for a given EDT.

We have developed a prototype tool that implements an
algorithm to solve the test generation problem by systematically
exploring the configurations of an mDSA corresponding to the given EDT,
$\Tt$.  %The tool does not handle timing.

We conducted experiments to validate the following:

\begin{enumerate}
\item For simple EDTs, the tool can find test cases succesfully.
\item There are some EDTs for which it can solve the test
  generation problem, where randomized techniques (likely) cannot.
\end{enumerate}

%\texttt{Random} uniformly samples a set of executions, $\Pi_R(\Tt)$, of size
%$N$, from the universal set of executions that have exactly $P$ inputs
%and checks whether $\Pi_R(\Tt)$ is a testsuite for $\Tt$. \texttt{RGRaF}, on
%the other hand, employs a heuristic to sample a set of executions of
%size $N$ and checks whether it is a testsuite. \texttt{RGRaF} takes a parameter
%$P$ to restrict the number of inputs in each execution.

First we look at a couple of small EDTs. Suppose we want a testcase for Row 1 in Table~\ref{tab:edt-alarm} from Section~\ref{sec:edt}. Encoding this in our tool gives a testcase (P; P) for Row 1, when default values for I and A are set to F.

%\begin{table}[h!]
%  \centering \def\arraystretch{1.2}
%  \caption{EDT for an Alarm module}
%  \label{tab:alarm-2}
%  \begin{tabular}{|c|c|c|c||c|c|}
%    \hline
%    sno & \specialcell{in \\ Ignition} &
%                                         \specialcell{in \\ PanicSw} &
%                                                                       \specialcell{in 
%    \\ Alarm} & \specialcell{out \\ Alarm} & 
%                                             \specialcell{out \\ Flash} \\
%    \hline 
%    1 & Off & (Press; Press)%\{$<1$s\} 
%    & Off & On &
%    \\
%    \hline
%
%    2 & On & & & Off & False \\
%    \hline
%  \end{tabular}
%  
%\end{table}

Next we look at Table~\ref{tab:wiper}, a partial representation of a wiper module in a car. The tool finds a testcase (park;notpark) for Row 2.

\begin{table}[h!]
  \centering \def\arraystretch{1.2}
  \caption{EDT for a Wiper module}
  \label{tab:wiper}
  \begin{tabular}{|c|c|c|c|c||c|c|}
    \hline
    sno & \specialcell{in \\ ignition} &
                                         \specialcell{in \\ wiperswitch} & 
                                                                       \specialcell{in 
    \\ parksensor} & \specialcell{in \\ error} & \specialcell{out \\ wipercmd} & 
                                             \specialcell{out \\ error} \\
    \hline 
    1 & on & on & 
    & false & wipe &
    \\
    \hline

    2 & & & (park;notpark) & & dontwipe & true \\
    \hline
  \end{tabular}
  
\end{table}


For the other claim, we created a specific kind of EDT for
which the probability of randomized techniques selecting a required
execution is very low. The EDT extends the three bit counter,
shown in Table \ref{tab:binary}, to five bits and adds a 
row to reset the counter. Our tool solves this problem, and hence for rows that have a very low probability of getting triggered
it is better than randomized techniques.

%Table \ref{tab:binary} shows a counter for $3$ bits.

\begin{table}
  \centering
  \renewcommand{\arraystretch}{1.2} 
  \caption{Implementing a binary counter for $3$ bits}
  \label{tab:binary}
  \begin{tabular}{|c|c|c|c|c||c|c|c|c|}
    \hline
    $T$ & $t_2$ & $t_1$ &$t_0$ & $S_T$ & $t_2$ & $t_1$ & $t_0$ & $S_T$                                                             
    \\
    \hline
     \checkmark & & & & & & & & $+$  \\
         
    \hline
     & & & $0$ & $+$ & & & $1$ & $-$  \\

    \hline
   & & $0$ & $1$ & $+$ & & $1$ & $0$ & $-$ \\

    \hline
   & $0$ & $1$ & $1$ & $+$ & $1$ & $0$ & $0$ & $-$ \\

    \hline
  \end{tabular}
\end{table}


%%% Local Variables:
%%% mode: latex
%%% TeX-master: "m"
%%% End:
