\subsection{Experiments}
\label{sec:experiments}
We have implemented a prototype test generator for EDTs using the translation of EDTs to mDSA-with-outputs \cite{mDSAcode}. In our implementation, we translate the given EDT $\Tt$ into an mDSA-with-outputs $\Bb^\Tt$ and simulate the semantics of $\Bb^\Tt$ using the finite transition system $\Ssf^{\Bb^\Tt}$ described in the proof of Theorem~\ref{thm:complexity}.
%A testsuite $\Pi(\Tt)$ for an EDT $\Tt$ is a set of executions such
%that for each row $R$ of $\Tt$ there exists an execution
%$\pi \in \Pi$ and $\pi$ triggers $R$. To find a testsuite, we need to solve the test generation problem for each row.
%The testsuite generation problem is to check whether there exists a testsuite for a given EDT.
%We have developed a prototype tool that implements an
%algorithm to solve the test generation problem by systematically
%exploring the configurations of an mDSA corresponding to the given EDT,
%$\Tt$.  %The tool does not handle timing.
%We conducted experiments to validate the following:
%\begin{enumerate}
%\item For simple EDTs, the tool can find test cases succesfully.
%\item There are some EDTs for which it can solve the test
%generation problem, where randomized techniques (likely) cannot.
%\end{enumerate}
%\texttt{Random} uniformly samples a set of executions, $\Pi_R(\Tt)$, of size
%$N$, from the universal set of executions that have exactly $P$ inputs
%and checks whether $\Pi_R(\Tt)$ is a testsuite for $\Tt$. \texttt{RGRaF}, on
%the other hand, employs a heuristic to sample a set of executions of
%size $N$ and checks whether it is a testsuite. \texttt{RGRaF} takes a parameter
%$P$ to restrict the number of inputs in each execution.
First we look at a couple of small EDTs. Suppose we want a testcase for Row 1 in Table~\ref{tab:edt-alarm} from Section~\ref{sec:edt}. Encoding this in our tool gives a testcase ($P; P$) for Row 1, when default values for $I$ and $A$ are set to $F$.
%\begin{table}[h!]
%  \centering \def\arraystretch{1.2}
%  \caption{EDT for an Alarm module}
%  \label{tab:alarm-2}
%  \begin{tabular}{|c|c|c|c||c|c|}
%    \hline
%    sno & \specialcell{in \\ Ignition} &
%                                         \specialcell{in \\ PanicSw} &
%                                                                       \specialcell{in 
%    \\ Alarm} & \specialcell{out \\ Alarm} & 
%                                             \specialcell{out \\ Flash} \\
%    \hline 
%    1 & Off & (Press; Press)%\{$<1$s\} 
%    & Off & On &
%    \\
%    \hline
%
%    2 & On & & & Off & False \\
%    \hline
%  \end{tabular}
%  
%\end{table}
Next we look at Table~\ref{tab:wiper}, a partial representation of a wiper module in a car. The tool finds a testcase (park;notpark) for Row 2.

\begin{table}[h!]
  \centering \def\arraystretch{1.1}
  \caption{EDT for a Wiper module}
  \label{tab:wiper}
  \begin{tabular}{|c|c|c|c|c||c|c|}
    \hline
    sno & \specialcell{in \\ ignition} &
                                         \specialcell{in \\ wiperswitch} & 
                                                                       \specialcell{in 
    \\ parksensor} & \specialcell{in \\ error} & \specialcell{out \\ wipercmd} & 
                                             \specialcell{out \\ error} \\
    \hline 
    1 & on & on & 
    & false & wipe &
    \\
    \hline

    2 & & & (park;notpark) & & dontwipe & true \\
    \hline
  \end{tabular}
  
\end{table}


We now present an EDT for which our tool can solve the test generation problem, whereas randomized techniques will have low success probability. %we created a specific kind of EDT for
%which the probability of randomized techniques selecting a required
%execution is very low. 
The EDT is a slight variation of the $N$ bit counter presented in Section~\ref{sec:mdsao-synt}. We depict a part of the EDT for three bits in
Table \ref{tab:binary}. %to five bits and adds a 
%row to reset the counter. The description of $N$-bit counters was presented in Section~\ref{sec:mdsao-synt}. 
In addition we add a row to check for all $t_i = 1$ , and also have an extra letter in the port $T$. Therefore there is exactly one sequence that makes all the $t_i$ equal to $1$. Randomized techniques have a low probability of generating such a test case. 
Our prototype implementation generates a test case for $N=5$. 

 %For the  and hence for rows that have a very low probability of getting triggered
%it is better than randomized techniques. 

%Table \ref{tab:binary} shows a counter for $3$ bits.

\begin{table}
  \centering
  \renewcommand{\arraystretch}{1.1} 
  \caption{Implementing a binary counter for $3$ bits}
  \label{tab:binary}
  \begin{tabular}{|c|c|c|c|c||c|c|c|c|}
    \hline
    $T$ & $t_2$ & $t_1$ &$t_0$ & $S_T$ & $t_2$ & $t_1$ & $t_0$ & $S_T$                                                             
    \\
    \hline
     \checkmark & & & & & & & & $+$  \\
         
    \hline
     & & & $0$ & $+$ & & & $1$ & $-$  \\

    \hline
   & & $0$ & $1$ & $+$ & & $1$ & $0$ & $-$ \\

    \hline
   & $0$ & $1$ & $1$ & $+$ & $1$ & $0$ & $0$ & $-$ \\

    \hline
  \end{tabular}
\end{table}

Our goal with the experiments was not to present a scalable tool for EDT test generation. We aimed to create an open source simulator for mDSA-with-outputs, which we have applied on EDT test case generation. As future work, we plan to investigate scalable algorithms for reachability in mDSA-with-outputs.

%%% Local Variables:
%%% mode: latex
%%% TeX-master: "m"
%%% End:
