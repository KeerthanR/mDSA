\newcommand{\lt}{\ell}
\newcommand{\rt}{r}

\section{Deterministic Suffix-reading Automata (DSA)}
\label{sec:prelims}

In this section, we recall the formal syntax and semantics of deterministic suffix-reading automata (DSA)~\cite{DBLP:journals/corr/abs-2410-22761}. For a finite alphabet $\Sigma$, we write $\Sigma^+$ for $\Sigma^* \setminus \{\epsilon\}$. 

\begin{definition}[Deterministic Suffix-reading Automata (DSA)]
Let $\Sigma$ be a finite alphabet. A DSA $\Aa$ is given by a tuple $(Q, \Sigma, q_0, \delta, F)$ where $Q$ is a finite set of states, $q_0$ is the initial state and $F \incl Q$ is a set of final states, and $\delta \incl Q \times \Sigma^+ \times \Sigma$, is a finite transition relation. Each transition therefore is of the form $(q, u, q')$ where $q, q' \in Q$ and $u \in \Sigma^+$. We write $\out(q) := \{ w \in \Sigma^+ \mid (q, u, q') \in \delta \}$. 
\end{definition}

A \emph{configuration} of a DSA is given by $(q, w)$ where $q$ is the current state, and $w$ is the word seen \emph{after reaching} $q$. The initial configuration is $(q_0, \epsilon)$. Transitions are as follows:
\begin{itemize}
\item $(q, w) \xra{a} (q, wa)$ if no string in $\out(q)$ is a suffix of $wa$,
\item $(q, w) \xra[u]{a} (q', \epsilon)$ if $u$ is the longest string of $\out(q)$ which is a suffix of $wa$, and $(q, u, q')$ is the corresponding transition. 
\end{itemize}
A configuration $(q, w)$ is accepting if $q \in F$ and $w = \epsilon$, signifying that the accepting state was reached on the last input.


We illustrate the semantics on an example. Consider the DSA in Figure~\ref{fig:dsa-examples} (b). Here is the sequence of configurations obtained on the word $bbaab$:
\begin{align*}
(q_0, \epsilon) \xra{b} (q_0, b) \xra{b} (q_0, b) \xra[ba]{a} (q_0, \epsilon) \xra{a} (q_0, a) \xra[ab]{b} (q_1, \epsilon)
\end{align*}

