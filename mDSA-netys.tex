\documentclass[runningheads,envcountsame]{llncs}

\usepackage{amssymb}
\usepackage{amsmath}
\usepackage{amsfonts}

\usepackage{hyperref}
\hypersetup{
    colorlinks=true,
    linkcolor=blue,
    filecolor=magenta,      
    urlcolor=cyan
    }

\usepackage{lineno}
\usepackage{tikz}
\usepackage{graphicx}
\usepackage{thmtools}
\usepackage{thm-restate}

\title{An automaton model to succinctly represent suffix-based specifications of a distributed system}

\titlerunning{An automaton model to succinctly represent suffix-based specifications of a distributed system}

\author{R Keerthan\inst{1,2} \and B Srivathsan\inst{2,3} \and
  R Venkatesh\inst{1}}

  \institute{Tata Consultancy Services - Innovation Labs, Pune \\
   \email{keerthanr@tcs.com, r.venky@tcs.com} \and Chennai Mathematical Institute,
  India \\
  \email{sri@cmi.ac.in} \and CNRS, ReLaX,
  IRL 2000, Siruseri, India }

  \begin{document}
  
  \maketitle

  \begin{abstract}
    Expressive Decision Table (EDT) is a formal notation developed by Venkatesh et al. (2014) for specifying requirements of a distributed reactive system. %The requirements typically check if a given stream of events last appeared in each component (in other words, a good suffix appears in each component), based on which specific outputs are generated. 
    EDTs have been successfully deployed in various industrial settings and several test generation tools for EDT specifications have been developed. However,  arguably, due to a complex interplay between concurrency and suffix-based event triggers, the EDT notation lacks a rigorous operational semantics. As a result, none of the test generation methods guarantees coverage completeness. When there is only a single component in the system (so, no concurrency), suffix-based event triggers can be conveniently captured using the recently introduced Deterministic Suffix-Reading Automata (DSA). %Being an automaton model, DSAs come equipped with precise operational semantics.
    %Recently, Deterministic Suffix-reading Automata (DSA) have been introduced to succinctly capture suffix-based specifications. % Transitions of a DSA are labeled with words -- intuitively, a transition is triggered at a state as soon as a stream of events ending with the label is seen. 

    In this work, we enrich DSAs to \emph{multi-port DSAs} --- a new automaton model which can succinctly represent \emph{concurrent} suffix-based specifications. %More specifically, we enrich transition labels to include a $\parallel$ operator. 
    %We call the resulting automata as \emph{multi-port DSAs}. 
    We describe a formal operational semantics for multi-port DSAs and show that EDTs are simply a subclass of multi-port DSAs. Consequently, test generation for EDTs naturally reduces to emptiness checking in multi-port DSAs. Finally, we prove that emptiness of multi-port DSAs is PSPACE-complete.
  \end{abstract}

  %Specifying requirements of a reactive system in a precise notation is an all important step before any kind of formal verification. The notation should be easy to use and also rigorous enough to enable automated analysis methods. 

  \section{Introduction}

  Why easy-to-write specifications are important? Why formal semantics is important? 
  
  EDT introduction. Some examples of EDTs. 

  An intricate EDT example for which no current tool can generate a test case? 

  Contributions

  2 to 3 pages

  \section{Preliminaries}

  Formal syntax, semantics and examples of DSA (2 pages)

  \section{Multi-port DSA}

  Multi-port alphabet, the new rules, automaton over these new rules, examples and operational semantics (3 pages)

  How multi-port DSA extends EDT and DSA (1 page)

  EDT Test generation as emptiness + explanation on the intricate example (1 page)

  \section{Complexity of emptiness}

  PSPACE upper bound (2 pages)

  PSPACE lower bound (2 pages)

  \section{Experiments? and Conclusion}

  \end{document}