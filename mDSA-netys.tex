\documentclass[runningheads,envcountsame]{llncs}

\usepackage{amssymb}
\usepackage{amsmath}
\usepackage{amsfonts}

\usepackage{hyperref}
\hypersetup{
    colorlinks=true,
    linkcolor=blue,
    filecolor=magenta,      
    urlcolor=cyan
    }

\usepackage{lineno}
\usepackage{tikz}
\usepackage{graphicx}
\usepackage{thmtools}
\usepackage{thm-restate}

\title{An automaton model to succinctly represent suffix-based specifications of a distributed system}

\titlerunning{An automaton model to succinctly represent suffix-based specifications of a distributed system}

\author{R Keerthan\inst{1,2} \and B Srivathsan\inst{2,3} \and
  R Venkatesh\inst{1}}

  \institute{Tata Consultancy Services - Innovation Labs, Pune \\
   \email{keerthanr@tcs.com, r.venky@tcs.com} \and Chennai Mathematical Institute,
  India \\
  \email{sri@cmi.ac.in} \and CNRS, ReLaX,
  IRL 2000, Siruseri, India }

  \begin{document}
  
  \maketitle

  \begin{abstract}
    Expressive Decision Table (EDT) is a formal notation developed by Venkatesh et al. (2014) for specifying requirements of a distributed reactive system. %The requirements typically check if a given stream of events last appeared in each component (in other words, a good suffix appears in each component), based on which specific outputs are generated. 
    EDTs have been successfully deployed in various industrial settings and several test generation tools for EDT specifications have been developed. However,  arguably, due to a complex interplay between concurrency and suffix-based event triggers, the EDT notation lacks a rigorous operational semantics. As a result, none of the test generation methods guarantees coverage completeness. When there is only a single component in the system (so, no concurrency), suffix-based event triggers can be conveniently captured using the recently introduced Deterministic Suffix-Reading Automata (DSA). %Being an automaton model, DSAs come equipped with precise operational semantics.
    %Recently, Deterministic Suffix-reading Automata (DSA) have been introduced to succinctly capture suffix-based specifications. % Transitions of a DSA are labeled with words -- intuitively, a transition is triggered at a state as soon as a stream of events ending with the label is seen. 

    In this work, we enrich DSAs to \emph{multi-port DSAs} --- a new automaton model which can succinctly represent \emph{concurrent} suffix-based specifications. %More specifically, we enrich transition labels to include a $\parallel$ operator. 
    %We call the resulting automata as \emph{multi-port DSAs}. 
    We describe a formal operational semantics for multi-port DSAs and show that EDTs are simply a subclass of multi-port DSAs. Consequently, test generation for EDTs naturally reduces to emptiness checking in multi-port DSAs. Finally, we prove that emptiness of multi-port DSAs is PSPACE-complete.
  \end{abstract}

  % We start with DSAs. Cannot handle concurrency well. We extend it here. As an application we give formal semantics to EDTs. Test generation algos have no theoretical guarantees. Now we give operational semantics.


  %Specifying requirements of a reactive system in a precise notation is an all important step before any kind of formal verification. The notation should be easy to use and also rigorous enough to enable automated analysis methods. 

  \section{Introduction}

  Why easy-to-write specifications are important? Why formal semantics is important? 
  
  EDT introduction. Some examples of EDTs. 

  An intricate EDT example for which no current tool can generate a test case? 

  Contributions

  2 to 3 pages

  \section{Preliminaries}

  Formal syntax, semantics and examples of DSA (2 pages)

  \section{Multi-port DSA}

  Multi-port alphabet, the new rules, automaton over these new rules, examples and operational semantics (3 pages)

  How multi-port DSA extends EDT and DSA (1 page)

  EDT Test generation as emptiness + explanation on the intricate example (1 page)

  \section{Complexity of emptiness}

  PSPACE upper bound (2 pages)

  PSPACE lower bound (2 pages)

  \section{Experiments? and Conclusion}

  
\section*{Parallel Operator Examples}

The parallel operator ($\|$) is indeed useful for showing that two conditions must both be satisfied but the order doesn't matter. Here are some examples where the parallel operator becomes increasingly advantageous with more inputs:

\subsection*{Car Security System}
For a car to start, you need: key in ignition ($K$), brake pedal pressed ($B$), transmission in park ($P$), and no alarm triggered ($A$).
\begin{itemize}
    \item Sequential: $K;B;P;A$ or any of the 24 possible permutations
    \item Parallel: $K\|B\|P\|A$ (much more compact)
\end{itemize}

\subsection*{Home Theater Setup}
To watch a movie, you need: TV on ($T$), sound system on ($S$), streaming device connected ($D$), internet working ($I$), and appropriate app opened ($A$).
\begin{itemize}
    \item Sequential: $T;S;D;I;A$ or any of the 120 possible permutations
    \item Parallel: $T\|S\|D\|I\|A$
\end{itemize}

\subsection*{Computer Boot Sequence}
For proper boot, you need: power supply working ($P$), motherboard operational ($M$), CPU functioning ($C$), RAM installed ($R$), storage device connected ($S$), and BIOS loaded ($B$).
\begin{itemize}
    \item Sequential: Would require one of 720 possible orderings
    \item Parallel: $P\|M\|C\|R\|S\|B$
\end{itemize}

\subsection*{Chemical Reaction Prerequisites}
For a particular reaction to occur, you need: correct temperature ($T$), proper pressure ($P$), catalyst present ($C$), reagent A ($A$), reagent B ($B$), and appropriate solvent ($S$).
\begin{itemize}
    \item Sequential: Would be one of 720 orderings
    \item Parallel: $T\|P\|C\|A\|B\|S$
\end{itemize}

\subsection*{Airport Security Checkpoint}
For a passenger to proceed, they need: boarding pass scanned ($B$), ID verified ($I$), security questions answered ($Q$), no prohibited items ($P$), and body scan cleared ($S$).
\begin{itemize}
    \item Sequential: One of 120 possible orderings
    \item Parallel: $B\|I\|Q\|P\|S$
\end{itemize}

The advantage of the parallel operator becomes even more apparent as the number of inputs increases, as the number of possible sequential combinations grows factorially ($n!$), while the parallel representation remains linear ($n$ inputs).


\section*{Complex Parallel Operation Examples}

\subsection*{Vehicle Security Override}
For security override: ignition on ($I$) and panic alarm pressed twice within a second ($P;P$) in any order.
\begin{itemize}
    \item Represented as: $I \parallel (P;P)$
\end{itemize}

\subsection*{Smartphone Lock Pattern}
To unlock a phone with pattern: fingerprint verification ($F$) and drawing specific gesture with three touch points ($T_1;T_2;T_3$).
\begin{itemize}
    \item Represented as: $F \parallel (T_1;T_2;T_3)$
\end{itemize}

\subsection*{Industrial Machine Safety Protocol}
To override emergency stop: supervisor key turned ($K$) and safety code entered as sequence of four button presses ($B_1;B_2;B_3;B_4$).
\begin{itemize}
    \item Represented as: $K \parallel (B_1;B_2;B_3;B_4)$
\end{itemize}

\subsection*{Banking Transaction Authorization}
For large transfer: account verification ($A$) and three-factor authentication with password, SMS code, and security question ($P;S;Q$).
\begin{itemize}
    \item Represented as: $A \parallel (P;S;Q)$
\end{itemize}

\subsection*{Nuclear Facility Operation}
To activate critical system: supervisor authorization ($S$), operator authorization ($O$), and three sequential safety checks ($C_1;C_2;C_3$).
\begin{itemize}
    \item Represented as: $S \parallel O \parallel (C_1;C_2;C_3)$
\end{itemize}

\subsection*{Flight Takeoff Clearance}
For takeoff permission: tower clearance ($T$), cabin crew ready ($C$), and pre-flight checklist with three mandatory sequential verifications ($V_1;V_2;V_3$).
\begin{itemize}
    \item Represented as: $T \parallel C \parallel (V_1;V_2;V_3)$
\end{itemize}

\subsection*{Smart Home Emergency Protocol}
To trigger house lockdown: security breach detected ($B$) and either owner verification ($O$) or emergency sequence of three panic button presses ($P_1;P_2;P_3$).
\begin{itemize}
    \item Represented as: $B \parallel (O \vee (P_1;P_2;P_3))$
\end{itemize}


  \end{document}
