
\section{Parallel Operator Examples}
\label{app:examples-of-parallel}
The parallel operator ($\|$) is indeed useful for showing that two conditions must both be satisfied but the order doesn't matter. Here are some examples where the parallel operator becomes increasingly advantageous with more inputs:

\subsection*{Car Security System}
For a car to start, you need: key in ignition ($K$), brake pedal pressed ($B$), transmission in park ($P$), and no alarm triggered ($A$).
\begin{itemize}
    \item Sequential: $K;B;P;A$ or any of the 24 possible permutations
    \item Parallel: $K\|B\|P\|A$ (much more compact)
\end{itemize}

\subsection*{Home Theater Setup}
To watch a movie, you need: TV on ($T$), sound system on ($S$), streaming device connected ($D$), internet working ($I$), and appropriate app opened ($A$).
\begin{itemize}
    \item Sequential: $T;S;D;I;A$ or any of the 120 possible permutations
    \item Parallel: $T\|S\|D\|I\|A$
\end{itemize}

\subsection*{Computer Boot Sequence}
For proper boot, you need: power supply working ($P$), motherboard operational ($M$), CPU functioning ($C$), RAM installed ($R$), storage device connected ($S$), and BIOS loaded ($B$).
\begin{itemize}
    \item Sequential: Would require one of 720 possible orderings
    \item Parallel: $P\|M\|C\|R\|S\|B$
\end{itemize}

\subsection*{Chemical Reaction Prerequisites}
For a particular reaction to occur, you need: correct temperature ($T$), proper pressure ($P$), catalyst present ($C$), reagent A ($A$), reagent B ($B$), and appropriate solvent ($S$).
\begin{itemize}
    \item Sequential: Would be one of 720 orderings
    \item Parallel: $T\|P\|C\|A\|B\|S$
\end{itemize}

\subsection*{Airport Security Checkpoint}
For a passenger to proceed, they need: boarding pass scanned ($B$), ID verified ($I$), security questions answered ($Q$), no prohibited items ($P$), and body scan cleared ($S$).
\begin{itemize}
    \item Sequential: One of 120 possible orderings
    \item Parallel: $B\|I\|Q\|P\|S$
\end{itemize}

The advantage of the parallel operator becomes even more apparent as the number of inputs increases, as the number of possible sequential combinations grows factorially ($n!$), while the parallel representation remains linear ($n$ inputs).


\section*{Complex Parallel Operation Examples}

\subsection*{Vehicle Security Override}
For security override: ignition on ($I$) and panic alarm pressed twice within a second ($P;P$) in any order.
\begin{itemize}
    \item Represented as: $I \parallel (P;P)$
\end{itemize}

\subsection*{Smartphone Lock Pattern}
To unlock a phone with pattern: fingerprint verification ($F$) and drawing specific gesture with three touch points ($T_1;T_2;T_3$).
\begin{itemize}
    \item Represented as: $F \parallel (T_1;T_2;T_3)$
\end{itemize}

\subsection*{Industrial Machine Safety Protocol}
To override emergency stop: supervisor key turned ($K$) and safety code entered as sequence of four button presses ($B_1;B_2;B_3;B_4$).
\begin{itemize}
    \item Represented as: $K \parallel (B_1;B_2;B_3;B_4)$
\end{itemize}

\subsection*{Banking Transaction Authorization}
For large transfer: account verification ($A$) and three-factor authentication with password, SMS code, and security question ($P;S;Q$).
\begin{itemize}
    \item Represented as: $A \parallel (P;S;Q)$
\end{itemize}

\subsection*{Nuclear Facility Operation}
To activate critical system: supervisor authorization ($S$), operator authorization ($O$), and three sequential safety checks ($C_1;C_2;C_3$).
\begin{itemize}
    \item Represented as: $S \parallel O \parallel (C_1;C_2;C_3)$
\end{itemize}

\subsection*{Flight Takeoff Clearance}
For takeoff permission: tower clearance ($T$), cabin crew ready ($C$), and pre-flight checklist with three mandatory sequential verifications ($V_1;V_2;V_3$).
\begin{itemize}
    \item Represented as: $T \parallel C \parallel (V_1;V_2;V_3)$
\end{itemize}

\subsection*{Smart Home Emergency Protocol}
To trigger house lockdown: security breach detected ($B$) and either owner verification ($O$) or emergency sequence of three panic button presses ($P_1;P_2;P_3$).
\begin{itemize}
    \item Represented as: $B \parallel (O \vee (P_1;P_2;P_3))$
\end{itemize}

