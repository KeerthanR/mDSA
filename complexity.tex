%\section{Complexity of emptiness}

\begin{theorem}\label{thm:complexity}
Emptiness for mDSA with outputs is PSPACE-complete.
\end{theorem}
\begin{proof}
    %\paragraph*{PSPACE upper bound.}
    \emph{$\PSPACE$ upper bound.} Let $\Bb$ be an mDSA-with-outputs. Consider the transition system $\Ss^\Bb$ describing the semantics of $\Bb$. Each node in the transition system is of the form $(q, w, \theta)$ where $w$ can become arbitrarily long. Therefore, the transition system, viewed as it is, is infinite. Suppose $M$ is the maximum length among all words used in the transition labels of $\Aa$. This is the maximum length of a suffix that will be checked for in a transition. To detect whether $u_1 \parallel u_2 \parallel \cdots \parallel u_k$ matches a configuration $(q, w, \theta)$, it is sufficient to maintain the last $M$ letters seen in each port. We now explain how this can be implemented.
    
    Assume that in configuration $(q, w, \theta)$, every projection $\proj{i}(w)$ of the word $w$ has at most $M$ letters. We apply the criteria for verifying if $(u_1 \parallel u_2 \parallel \cdots \parallel u_k)$ matches the tape configuration $(w, \theta)$ as in Definition~\ref{def:match}. Now, when we append $a \in \Sigma_i$ to the end of $w$: if $wa$ contains $M+1$ letters in port $i$, that is $|\proj{i}(wa)| = M+1$, then let $w'$ be the word obtained by deleting from $wa$ the first occurrence of a letter in $\Sigma_i$. Let $b$ be this letter. If $b$ appeared at a position strictly greater than $\theta$ in $wa$, then we do not need to change the tape head ($\theta' = \theta$); otherwise, $b$ appears on or before $\theta$, and we reduce tape position by $1$ ($\theta' = \theta - 1$). This gives a new tape configuration $(w', \theta')$. Transitions point to the new truncated configuration. Call the new truncated transition system as $\Ssf^\Bb$.
    
    
    This bounds the tape length to $M \times k$, where $k$ is the total number of ports. So, the total size of the truncated transition system is exponential in the size of the input. To verify emptiness, it is sufficient to guess a path in this transition system. Since each node requires polynomial space, and the length of the witness is bounded by an exponential in the size of the input, we deduce a $\PSPACE$ upper bound for the emptiness problem, following standard results from complexity theory.

    \emph{$\PSPACE$-hardness.} For the hardness, we give a reduction from this problem: given DFAs $D_1, D_2, \dots, D_n$ over a common alphabet $\Sigma$, the DFA-intersection-emptiness asks if there exists a word in the language of all the $n$ DFAs. This is known to be $\PSPACE$-complete~\cite{10.1109/SFCS.1977.16}. Let $D_j = (Q_j, q^{init}_j, \delta_j, F_j)$ be the description of DFA $D_j$. The mDSA-with-outputs $\Bb^D$ that we will construct makes use of $n$ ports to store the state reached by each of the DFAs.

    We will have one input port $I$ with $\Sigma$ as alphabet, and $n$ ports $S_j$ (for $j \in \{1, \dots, n\}$), with $\Sigma_j = Q_j$ being the states of DFA $D_j$. We add a special port $L$ with alphabet $\$$ that will be used for book-keeping. Let $s^{init}$ be the initial state of $\Bb^D$. There is an initial transition that reads the special character $\$$ and writes the initial state of each DFA to the respective ports, and moves to $s_0$:
    \begin{align*}
    s^{init}~~ \xrightarrow[~S_1: q^{init}_1 ~\parallel~ S_2: q^{init}_2 ~\parallel~ \cdots ~\parallel~ S_n: q^{init}_n~]{L: \$} ~~s_0
    \end{align*}
    From $s_0$ we have a gadget that reads the next input and then sequentially updates the states of each DFA by writing the new states in the respective ports. 
    \begin{itemize}
        \item the process starts with a transition $s_0~\xra[L:\$]{I:a} s^a_1$; 
        \item there are states $s^a_{j}$ for all $j \in \{1, \dots, n\}$, and for all $a \in \Sigma$;
        \item for every transition $(q_j, a, q'_j)$ in $D_j$ we have: $s^a_j \xra[~L:\$~\parallel~S_j: q'_j~]{L:\$ ~\parallel~ S_j: q_j~} s^a_{j+1}$, with $s^a_{n+1} := s_0$.
    \end{itemize}
    When an $a$ is read at $s_0$, the automaton moves to $s^a_1$ and the tape head moves to the position of $a$. Since the same transition also writes a $\$$ to $L$, there will be an outgoing transition from $s^a_1$ that matches: this transition would be the one corresponding to the current state of $D_1$. Then, the automaton goes to $s^a_2$, and the process continues until the automaton comes back to $s_0$, giving rise to a sequence of $n$ $\epsilon$-transitions from $s^a_1$ back to $s_0$. 

    This is the basic construction that simulates the $n$ DFAs using an mDSA-with-outputs. To detect acceptance, we can add special ports $B_1, B_2, \dots, B_n$. Whenever we reach an accepting state in $Q_j$ we write $1$ to this port, and when we reach a non-accepting state, we write $1$. Add a transition $B_1 = 1 \parallel B_2 = 1 \parallel \cdots \parallel B_n = 1$ from $s_0$ to a special state $f$. The intersection of $D_1, \dots, D_n$ is empty iff $f$ is reachable.

    For every letter $a$ and every port $j$ there is a state $s^a_j$. There are transitions $s^a_j$ to $s^a_{j+1}$ for each every transition in $D_j$. Total number of states is $(|\Sigma| \cdot n) + 2$, and the total number of transitions is proportional to $\sum_j \delta_j$.  Therefore, the overall construction is polynomial-time.


\end{proof}