%\section{Complexity of emptiness}

\begin{theorem}
Emptiness for mDSA with outputs is PSPACE-complete.
\end{theorem}
\begin{proof}
    Let $\Aa$ be an mDSA with outputs. Consider the transition system $S^\Aa$ describing the semantics of $\Aa$. Each node in the transition system is of the form $(q, T, i_1, i_2, \dots, i_p)$ where $T$ is the input tape, which can become arbitrarily long. Therefore, the transition system, viewed as it is, is infinite. Suppose $M$ is the maximum length among all words used in the transition labels of $\Aa$. To detect whether a transition $\lt: u_1 \parallel u_2 \parallel \cdots \parallel u_k$ matches a configuration, it is sufficient to maintain $M$ letters in each port, after each tape head $\lt$. Hence, letters which are not relevant anymore can be deleted from the tape. This bounds the tape length to $M \times p \times k$, where $k$ is the total number of ports. So, the total size of the truncated transition system is exponential in the size of the input. To verify emptiness, it is sufficient to guess a path in this transition system. Since each node requires polynomial space, and the length of the witness is bounded by an exponential in the size of the input, we deduce a PSPACE upper bound for the emptiness problem (following standard results from complexity theory).

    For the hardness, we reduce the problem of DFA-intersection-emptiness  to emptiness of mDSA with outputs. Given DFAs $D_1, D_2, \dots, D_n$ over a common alphabet $\Sigma$, the DFA-intersection-emptiness asks if there is a word in the language of all the $n$ DFAs. This is PSPACE-complete. Let $D_j = (Q_j, q^{init}_j, \delta_j, F_j)$ be the description of DFA $D_j$. The mDSA $\Aa^D$ that we will construct makes use of IO variables to store the state reached by each of the DFAs.

    We will have one input port with $\Sigma$ as alphabet, and $n$ IO ports, with $\Sigma^{io}_j = Q_j$ with alphabet being the states of DFA $D_j$.

    The initial state of $\Aa^D$, called $s^{init}$, is a write state. There is an initial transition that writes the initial state to the respective ports and moves to  a read state $s_0$:
    \begin{align*}
    s^{init} \xrightarrow[q^{init}_1 \parallel q^{init}_2 \parallel \cdots \parallel q^{init}_n]{} s_1
    \end{align*}

    From $s_0$ there is a transition on every $a \in \Sigma$, followed by a transition sequence that will update the IO ports to correspond to the new states reached by each DFA. Suppose $(q_1, q_2, \dots, q_n)$ is the current configuration of the DFAs $D_1, D_2, \dots, D_n$ maintained as the last seen value in the corresponding ports. For a letter $a \in \Sigma$, suppose $(q_j, a, q'_j)$ are the respective transitions in each DFA. There is a sequence in $\Aa^D$ as follows.
    \begin{align*}
    s_1 \xra{a \parallel Q_1 = q_1} r^{q_1}_1 \xra[Q_1 := q'_1]{} s_2 \xra{a \parallel Q_2 = q_2} r^{q_2}_2 \xra[Q_2 := q'_2]{} s_3 \cdots s_{n-1} \xra{a \parallel Q_n = q_n} r^{q_{n-1}}_{n-1} \xra[Q_n := q'_n]{} s_0 
    \end{align*}
    States $s_0, s_1, \dots, s_{n-1}$ are read states and $r_0, r_1, \dots, r_{n-1}$ are write states. State $s_j$ reads the current state of $D_j$ and the input letter, and state $r_j$ writes the new target state into the port. We assume that there are $n$ tape heads $1, \dots, n$, with $s_j, r_j$ using tape head $\lt_{j+1}$. 

    Hence, for every letter $a$, there are transitions from $s_1$ to $r^{q}_1$, for each $q \in Q$. All these write states merge to state $s_2$, and so on. Finally they come back to $s_1$. Total number of states is $(n + 1) + \sum{j} |Q_j|$. Each transition contains only letters from a port. Therefore, the overall construction is polynomial-time.
    
    To detect acceptance, we can add special IO ports $B_1, B_2, \dots, B_n$. Whenever we reach an accepting state in $Q_j$ we write $1$ to this port, and when we reach a non-accepting state, we write $1$. Add a transition $B_1 = 1 \parallel B_2 = 1 \parallel \cdots \parallel B_n = 1$ from $s_1$ to a special state $f$. The intersection of $D_1, \dots, D_n$ is empty iff $f$ is reachable in $\Aa^D$. 


\end{proof}