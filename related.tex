\section{Related Work}
Use of automata to model distributed systems has been explored widely.
Esterel~\cite{DBLP:journals/scp/BerryG92} is a synchronous programming language
that provides a parallel composition operator. Harel's
Statecharts!\cite{DBLP:journals/scp/Harel87} also support parallel composition
of state transition systems. In both these languages the reaction of a system
in response to inputs on each port can be specified separately and the
different specifications can be composed using the parallel operator. A system
model can then refer to the states of each port's specification to model the
system behaviour. mDSAs specifications are more succinct than Statecharts or Esterel specifications of System behaviours similar to the examples described in this
paper. Petri nets~\cite{CAPetri}
have been studied extensively to model concurrent systems. In Petri nets each
letter of each port's alphabet will have to be represented using a place and
the places combined to model the system. This will not be as succinct as an
mDSA. 
The work that is closest to ours is Input/Output Partial Order
Automata(IOPOA)~\cite{10.5555/2391293.2391305}. In their work transitions
execute non-atomically reacting to asynchronous inputs on several ports. The
main differences between IOPOA and mDSA are - in mDSA transitions react
atomically, transition labels can be strings for each port and not just letters
and different transitions may trigger on the same string at a port. For example consider the first mDSA in Figure~\ref{fig:mdsa-examples}, the transition from $Off$ to $Started$ and $Started$ to $Drive$ can be taken on the same input $B$. In an IOPOA this will require two occurences of $B$. IOPOA was primarily designed to enable test generation whereas the aim of mDSAs is succinct specification of requirements.
