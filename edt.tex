

\section{Expressive decision tables}
\label{sec:edt}
Expressive Decision Tables (EDTs)~\cite{DBLP:conf/date/VenkateshSKA14} are a tabular
notation for specifying software requirements and have been used in various
industrial settings~\cite{DBLP:conf/enase/VenkateshSZA15a,DBLP:conf/icst/AgrawalVSZV20}.  An EDT specification consists
of tables where columns specify input, output and input/output (I/O) ports, and rows specify the relationship between input and output
values on these ports. Each port is associated with an alphabet and each cell in the table consists of a string of letters from this alphabet.% with discrete
The original work on EDTs contained timing constraints in the cells and interpreted the EDT over a notion of discrete-time. In this paper we consider EDT without time. A cell in  an EDT table
is said to match when the string given in that cell is a suffix of the values seen for the
input port corresponding to that cell's column. A row of an EDT matches when
all the cells corresponding to that row match and one cell matches on the last
input.  Table~\ref{tab:edt-alarm} gives the EDT specification of the Car Alarm module described in the beginning of Section~\ref{sec:multiport-outputs}. We write $F$ for \texttt{On} and \texttt{False}, $P$ for \texttt{Press} and $R$ for \texttt{Release}. %an example, adapted from
%\cite{DBLP:conf/enase/VenkateshSZA15a,DBLP:conf/icst/AgrawalVSZV20}, specifying a subset of requirements of a car alarm
%module.

\newcolumntype{A}{>{\centering}p{1cm}}
\begin{table}[h!]
  \centering \def\arraystretch{1.5}
  \caption{EDT for an Alarm module}
  \label{tab:edt-alarm}
	\begin{tabular}{|A|A|A|A||A|c|}
    \hline
    s.no & in  I & in  P & in A & out  A & out  F \\
	 \hline
    1 & F & P;P & F& N & \\
    \hline
    2 & & R;R;R & N & F & F \\
    \hline
    3 & N & & & F & F \\
    \hline
  \end{tabular}
  
\end{table}
In the example EDT the input ports are prefixed with 
\emph{in} and output ports are prefixed with \emph{out}. Input/Output ports are prefixed with both \emph{in} and \emph{out}. These are the ports that appear on both sides of the EDT, as inputs as well as outputs. 
%The EDT of
%Table~\ref{tab:edt-alarm} specifies the
%following requirements:
%\begin{enumerate}
%	\item If the most recent values on Ports \emph{I} (Ignition) and \emph{A} (Alarm) are \emph{F}(Off), and
%		the last two values on the Port \emph{P} are $P;P$ (Press), then
%		the value \emph{N}(On) is output on the port \emph{A}.

%\item If the last three values on Port \emph{P} are
%	$R;R;R$ (Release), and the last value for \emph{A} is \emph{N}, then output the value \emph{F} (False) on \emph{F}
%		and output \emph{F} on \emph{A}.

%	\item If last value on \emph{I} is \emph{N}, then \emph{F} is output on \emph{F} 
%		and \emph{F} is output on \emph{A}.
%\end{enumerate}

A \emph{test case} for a requirement is a sequence of values  that match the
EDT row corresponding to that requirement. The use of EDTs and test case generation
algorithms from EDT specifications have been widely studied.
~\cite{DBLP:conf/enase/VenkateshSZA15a,DBLP:conf/icst/AgrawalVSZV20}. However, the notation lacks a rigorous
formal semantics, except for a brief description of key elements of formal
semantics in~\cite{DBLP:conf/date/VenkateshSKA14}. Hence, none of the test case algorithms use
formal techniques like model checking and instead rely on random generation
using heuristics. These algorithms, therefore, cannot prove that an EDT row
cannot be matched and also find it hard to match certain rows. We overcome both
these limitations by providing a formal semantics for EDTs (without timing)
through a translation to mDSA-with-outputs.
%When certain requirements do not get covered, a manual 
% As we will show in this paper, the mechanics of
%EDTs is intricate due to various interactions between the rows. An
%automated tool to generate test cases or \emph{guarantee}
%non-coverability is indispensable. To achieve this, the first step is
%to have a formal operational semantics for EDT.

\subsection{Translating EDT to mDSA-with-outputs}

An EDT with $C$ columns and $R$ rows is translated to an mDSA, with one
state $q_0$ and an alphabet $\Sigma = \langle \Sigma_1 \cdots \Sigma_{|C|} \rangle$.
Each $\Sigma_i$ corresponds to the port of column $i$ and the
letters of the alphabet correspond to the values that the signal can take,
prefixed by $\langle Portname: \rangle$. 
Thus if the Port name of the first
column is $A$ and its alphabet is ${ a_1, a_2, \cdots, a_k}$ then the
corresponding alphabet $\Sigma_1 = {A:a_1, A:a_2 \cdots A:a_k}$. 
Each EDT row is translated to an mDSA as follows:
\begin{itemize}
\item The mDSA has exactly one state $q_0$
%\item 
%The alphabets of all the ports together make a multi-port alphabet for the mDSA
%\item
%Each port has a corresponding tape. Each row has a tape head in each tape, corresponding %to itself.
\item
There is a transition corresponding to each row with $q_0$ as both the source and target state. 
\item
The label of each transition consists of all the strings in the cells of the input columns of the row combined using the $\parallel$ operator.
\item 
The output of the transitions are the strings in the cell corresponding to the output columns, combined using the $\parallel$ operator.
\end{itemize}

The mDSA in Figure~\ref{fig:state-diagram} is a mDSA specification corresponding the EDT given in Table~\ref{tab:edt-alarm}

\begin{figure}[htbp]
    \centering
\begin{tikzpicture}[
    > = stealth,
    shorten > = 1pt,
    auto,
    node distance = 3cm,
    semithick
]

% Setup the styles for the states
\tikzstyle{accepting}=[double, draw=black, circle, minimum size=1cm]
\tikzstyle{state}=[draw=black, circle, minimum size=1cm]

% Draw the state
\node[state, initial, accepting] (q0) {$q_0$};

% Draw the transition (self-loop)
\draw (q0) edge[loop above] node{$I:F$$\|$$P:P;P:P$$\|$$A:F$/$A:N$} (q0);
\draw (q0) edge[loop right] node{$P:R;P:R;P:R$$\|$$A:N$/$A:F$$\|$$F:F$} (q0);
\draw (q0) edge[loop below] node{$I:N$/$A:F$$\|$$F:F$} (q0);

\end{tikzpicture}
\caption{mDSA of the Alarm EDT}
    \label{fig:state-diagram}
\end{figure}

To generate a test for a certain row $r$, we can first add a special state $q_r$, and make the transition corresponding to this row point to $q_r$ instead of $q_0$. A test  for $r$ then reduces to a witness for the reachability of $q_r$. 

%%The mDSA above is a succinct representation of the EDT semantics. The states of a multi-port DSA are defined as per the requirements of the system it models. For a system with internal variables $S_1,S_2,\dots$ with each variable $S_i$ having an alphabet $\Sigma_{S_i}$, the corresponding \mdsa has states $Q=\Sigma_{S_1} \times \Sigma_{S_2} \times \dots$, but as seen earlier, this can be captured using additional tape instead. The additional tapes $B_{S_1}, B_{S_2}, \dots$ are for the internal variables used in the system. These variables can read input as well as produce output, over the alphabet representing their state space. %For example, a variable $S_i$ has an alphabet $\Sigma_{S_i}$ with letters that can be input as well as output.
%%
%%Transitions must then be enhanced in our semantics i.e. a transition can now require the local variables $S_1, S_2,\dots$ to have values $(s_1, s_2 \dots)$  for it to match. When triggered, the values then change to $(s'_1,s'_2,\dots)$. Assume each variable $S_i$ has its corresponding tape labeled $B_{S_i}$%(we relabel the tapes $B_{k+1}, B_{k+2}, \dots, B_{k'}$ appropriately)
%%. Let us use the syntax $\langle (s_1, s_2 \dots), (u_1, \dots, u_k), (s'_1,s'_2,\dots) \rangle$ to represent this.
%%
%%\begin{enumerate}
%%\item A transition (with label $j$) $\langle (s_1, s_2 \dots), (u_1, \dots, u_k), (s'_1,s'_2,\dots) \rangle$ is matched if each $B_i (i\le k)$ has $u_i$ as suffix, the last letter of each $B_{S_i}$ is $s_i$, and either $H_{i}^{j}$ is at the end or the entire $u_i$ is after $H_{i}^{j}$. Additionally, at least one tape is non-empty after $H_{i}^{j}$.
%%
%%\item A matched transition is then triggered, moving each $H_{i}^{j}$-head of $B_i$ (or $B_{S_i}$) to the end. Additionally $\forall s'_i \ne \epsilon$, the corresponding $B_{S_i}$ extends by one position to the right and stores $s'_i$.
%%\end{enumerate}
%%
%%\textcolor{red}{Semantics for Row sequences}
%%We introduce the notion of a multi-transition, a transition with multiple parts that must occur in sequence. Let us illustrate with a two-part transition $\langle (s_1, s_2 \dots), (u_1, \dots, u_k), (s'_1,s'_2,\dots)$ ; $(s''_1, s''_2 \dots), (u'_1, \dots, u'_k), (s'''_1,s'''_2,\dots) \rangle$. For it to match, each $B_i (i\le k)$ must have $u_i  u'_i$ as suffix, each $B_{S_i}$ has $s_i  s''_i$, and either $H_{i}^{j}$ is at the end or the entire $u'_i$ occurs after $H_i^j$. Additionally, at least one tape is non-empty after $H_i^j$ heads. Not only this, all of $(s''_1, s''_2 \dots), (u'_1, \dots, u'_k)$ must occur after $(s_1, s_2 \dots), (u_1, \dots, u_k)$. That is each for any $i,j$ we have $u'_i$ and $s''_i$ appearing after $u_j$ and $s_j$. When the matched transition is triggered, the tapes and their head positions are updated as earlier.
%%
%%\textcolor{red}{TODO: Reachability for mDSA vs Test generation for EDT. Complexity result (possibly PSPACE-complete)}
%%
%\input{basic.tex}


%%% Local Variables:
%%% mode: latex
%%% TeX-master: "mDSA"
%%% End:
