

\section{Expressive decision tables}
\label{sec:edt}
Expressive Decision Tables (EDTs)~\cite{DBLP:conf/date/VenkateshSKA14} are a tabular
notation for specifying software requirements and have been used in various
industrial settings~\cite{DBLP:conf/enase/VenkateshSZA15a,DBLP:conf/icst/AgrawalVSZV20}.  An EDT specification consists
of tables where columns specify input, output and input/output (I/O) ports, and rows specify the relationship between input and output
values on these ports. Each port is associated with an alphabet and each cell in the table consists of a string of letters from this alphabet.% with discrete
The original work on EDTs contained timing constraints in the cells and interpreted the EDT over a notion of discrete-time. In this paper we consider EDT without time. A cell in  an EDT table
is said to match when the string given in that cell is a suffix of the values seen for the
input port corresponding to that cell's column. A row of an EDT matches when
all the cells corresponding to that row match and one cell matches on the last
input.  Table~\ref{tab:edt-alarm} gives the EDT specification of the Car Alarm module described in the beginning of Section~\ref{sec:multiport-outputs}. We write $F$ for \texttt{Off} and \texttt{False}, $N$ for \texttt{On}, $P$ for \texttt{Press} and $R$ for \texttt{Release}.

\newcolumntype{A}{>{\centering}p{1cm}}
\begin{table}[h!]
  \centering \def\arraystretch{1.5}
  \caption{EDT for an Alarm module}
  \label{tab:edt-alarm}
	\begin{tabular}{|A|A|A|A||A|c|}
    \hline
    s.no & in  I & in  P & in A & out  A & out  F \\
	 \hline
    1 & F & P;P & F& N & \\
    \hline
    2 & & R;R;R & N & F & F \\
    \hline
    3 & N & & & F & F \\
    \hline
  \end{tabular}
  
\end{table}
In the example EDT the input ports are prefixed with 
\emph{in} and output ports are prefixed with \emph{out}. Input/Output ports are prefixed with both \emph{in} and \emph{out}. These are the ports that appear on both sides of the EDT, as inputs as well as outputs. 

A \emph{test case} for a requirement is a sequence of values  that match the
EDT row corresponding to that requirement. The use of EDTs and test case generation
algorithms from EDT specifications have been widely studied.
~\cite{DBLP:conf/enase/VenkateshSZA15a,DBLP:conf/icst/AgrawalVSZV20}. However, the notation lacks a rigorous
formal semantics, except for a brief description of key elements of formal
semantics in~\cite{DBLP:conf/date/VenkateshSKA14}. Hence, none of the test case algorithms use
formal techniques like model checking and instead rely on random generation
using heuristics. These algorithms, therefore, cannot prove that an EDT row
cannot be matched and also find it hard to match certain rows. We overcome both
these limitations by providing a formal semantics for EDTs (without timing)
through a translation to mDSA-with-outputs.


\subsection{Translating EDT to mDSA-with-outputs}

An EDT with $C$ columns and $R$ rows is translated to an mDSA, with one
state $q_0$ and an alphabet $\Sigma = \langle \Sigma_1 \cdots \Sigma_{|C|} \rangle$.
Each $\Sigma_i$ corresponds to the port of column $i$ and the
letters of the alphabet correspond to the values that the signal can take,
prefixed by $\langle Portname: \rangle$. 
Thus if the Port name of the first
column is $A$ and its alphabet is ${ a_1, a_2, \cdots, a_k}$ then the
corresponding alphabet $\Sigma_1 = {A:a_1, A:a_2 \cdots A:a_k}$. 
Each EDT row is translated to an mDSA as follows:
\begin{itemize}
\item The mDSA has exactly one state $q_0$

\item
There is a transition corresponding to each row with $q_0$ as both the source and target state. 
\item
The label of each transition consists of all the strings in the cells of the input columns of the row combined using the $\parallel$ operator.
\item 
The output of the transitions are the strings in the cell corresponding to the output columns, combined using the $\parallel$ operator.
\end{itemize}

The mDSA in Figure~\ref{fig:state-diagram} is an mDSA specification corresponding the EDT given in Table~\ref{tab:edt-alarm}

\begin{figure}
    \centering
\begin{tikzpicture}[
    > = stealth,
    shorten > = 1pt,
    auto,
    node distance = 3cm,
    semithick
]

% Setup the styles for the states
\tikzstyle{accepting}=[double, draw=black, circle, minimum size=1cm]
\tikzstyle{state}=[draw=black, circle, minimum size=1cm]

% Draw the state
\node[state, initial, accepting] (q0) {$q_0$};

% Draw the transition (self-loop)
\draw (q0) edge[loop above] node{$I:F$$\|$$P:P;P:P$$\|$$A:F$/$A:N$} (q0);
\draw (q0) edge[loop right] node{$P:R;P:R;P:R$$\|$$A:N$/$A:F$$\|$$F:F$} (q0);
\draw (q0) edge[loop below] node{$I:N$/$A:F$$\|$$F:F$} (q0);

\end{tikzpicture}
\caption{mDSA of the Alarm EDT}
    \label{fig:state-diagram}
\end{figure}

To generate a test for a certain row $r$, we can first add a special state $q_r$, and make the transition corresponding to this row point to $q_r$ instead of $q_0$. A test  for $r$ then reduces to a witness for the reachability of $q_r$. 

