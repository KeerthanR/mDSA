\section{Extending DSA with multiple ports}

Consider a special kind of an alphabet
$\Sigma = \langle \Sigma_1, \Sigma_2, \dots, \Sigma_k \rangle$ such
that $\Sigma_i \cap \Sigma_j = \emptyset$ for all $i, j \in
[1..k]$. We will call $\Sigma_i$ as the alphabet of \emph{port} $i$,
and $\Sigma$ as a \emph{multi-port} alphabet. We
look at DFAs over such multi-port alphabets. Such DFAs model systems
that listen to inputs from different processes and perform actions
based on them. Sometimes the order in which the system receives its
inputs from different ports is not relevant.%
  
For example, at a state
$s$, if the system receives $a$ from port 1 and $b$ from port 2, in any
order, then it has to go to state $t$.  A DFA would model this with
transitions: $s \xra{a} s_a \xra{b} t$ and $s \xra{b} s_b \xra{a}
t$. A DSA would contain two transitions $s \xra{ab, ba} t$ (and some
other transitions, if needed, to take care of $aa$, $bb$). To get a
more succinct notation we will use a $\parallel$ operator. We will
write $s \xra{a \parallel b} t$ to mean that at state $s$, when
both $a$ and $b$ are received, the automaton moves to $t$. When there
are several components, this notation leads to significant
succinctness, for instance $a_1 \parallel a_2 \cdots \parallel a_n$
stands for all the $n!$ permutations of $a_1$ to $a_n$. We will also
allow expressions of the form $a_1 a_2 \parallel b$, which stands for
the set of words $\{a_1 a_2 b, a_1 b a_2, b a_1 a_2\}$ which shuffles
$a_1a_2$ and $b$. We will now formalize this idea by enhancing DSA with
the $\parallel$ operator and then consider the problem of synthesizing
such extended DSA. We begin with some notation.%




% \paragraph*{Notation} For a word $w \in \Sigma^*$, we write
% $\proj{i}(w)$ for the projection of $w$ onto the set $\Sigma_i$. For
% instance, if $\Sigma_1 = \{a_1, a_2\}, \Sigma_2 = \{b_1, b_2\}$ and
% $w = a_1 b_1 a_2 a_1 b_2 b_2$, we have $\proj{1}(w) = a_1 a_2 a_1$ and
% $\proj{2}{w} = b_1 b_2 b_2$.  We write $\partial w$ for the $k$-tuple
% $(\proj{1}(w), \proj{2}(w), \dots, \proj{k}(w))$ of projections of $w$
% onto each port $\Sigma_i$, and will call $\partial w$ the
% \emph{decomposition} of $w$. Notice that if
% $\partial w_1 = \partial w_2$ for two words $w_1, w_2$, then $w_1$ and
% $w_2$ have the same order of events within each port, but could have a
% different ordering between letters from different ports.%


\begin{definition} Let
 $\Sigma = \langle \Sigma_1, \Sigma_2, \dots, \Sigma_k \rangle$ be a
 multi-port alphabet. A \emph{multi-port (deterministic)
   suffix-reading automaton} (written in short) $\Aa$ is a
 tuple $(Q, \Sigma, q^{init}, \delta, F)$ where $Q$, $q^{init}$ and
 $F$ are a finite set of states, the initial state and a set of
 accept states, respectively. The transition relation
 $\delta \incl Q \times (\Sigma_1^* \times \Sigma_2^* \times \cdots
 \times \Sigma_k^*) \times Q$: each transition is of the form
 $(q, (u_1 \parallel u_2 \parallel \cdots \parallel u_k), q')$ where
 $u_i \in \Sigma_i^*$ (not all of them can be $\epsilon$).
 % We assume that for every two outgoing
 % transitions
 % $(q, (u_1, u_2, \dots, u_k), q_1)$ and
 % $(q, (v_1, v_2, \dots, v_k), q_2)$ there is at least one port $i$
 % such that $u_i$ and $v_i$ are suffix-incomparable, that is,
 % $u_i \not \sfx v_i$ and $v_i \not \sfx u_i$.
\end{definition}%


A configuration is given by $(q, i, \Lambda)$ where $\Lambda$ gives the position of the left and right tape heads for every tape $j$: $\Lambda_j(\lt), \Lambda_j(\rt)$.

$(q, i, \Lambda) \xra{T(i+1)} (q', i+1, \Lambda') $ as follows: (let $a = T(i+1)$)
\begin{itemize}
\item if $a \in \Sigma_m$, then $a$ is appended to the right of tape $B_m$, and the tape head $\rt_m$ moves one step to the right.
\item there is some transition $(q, u_1 \parallel u_2 \parallel \cdots \parallel u_k) q'$ such that each $u_i$ is a suffix of $B_i$ and there is at least one $i$ such that $u_i$ is a suffix of $B_i[\Lambda(\lt), \Lambda(\rt)]$,
\item $\Lambda'_i(\lt) = \Lambda'_i(\rt) = \Lambda_i(\rt) + 1$. 
\end{itemize}

\begin{definition}[Multi-port DSA (mDSA)]
    Let $\Sigma = (\Sigma_1, \Sigma_2, \dots, \Sigma_k)$ be a multi-port alphabet. A multi-port DSA (mDSA in short) $\Aa$ is given by a tuple $(Q, \Sigma, q_0, \Delta, F)$ where $Q$ is a finite set of states, $q_0$ is the initial state and $F \incl Q$ is a set of final states, and $\Delta : Q \times \Sigma_1^* \times \Sigma_2^* \times \cdots \times \Sigma_k^* \rightharpoonup Q$ is a  partial function with a finite domain. We assume that  Each transition therefore is of the form $(q, u, q')$ where $q, q' \in Q$ and $u \in \Sigma^*$. 
    
    We write $\out(q) := \{ w \in \Sigma^+ \mid \Delta(q, u) \text{ is defined }\}$.
    \end{definition}

    There is an input read-only tape  and $k$ work tapes, one for each port. The mDSA reads the input one by one. On reading a letter, it copies it to the corresponding work tape. The right head of the work tape moves to this position. If there is some rule in $\out(q)$ which is matched: $u_i$ appears as a suffix of correpsonding tape, and there is at least one tape such that $u_i$ appears as a suffix of $T_i[\lt_i, \rt_i]$. This corresponds to the examples of the introduction. Explain.

