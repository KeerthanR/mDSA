\label{sec:multi-port-inputs}%


Consider a special kind of an alphabet
$\Sigma = \langle \Sigma_1, \Sigma_2, \dots, \Sigma_k \rangle$ such
that $\Sigma_i \cap \Sigma_j = \emptyset$ for all $i, j \in
[1..k]$. We will call $\Sigma_i$ as the alphabet of \emph{port} $i$,
and $\Sigma$ as a \emph{multi-port} alphabet. We
look at DFAs over such multi-port alphabets. Such DFAs model systems
that listen to inputs from different processes and perform actions
based on them. Sometimes the order in which the system receives its
inputs from different ports is not relevant.%
  
For example, at a state
$s$, if the system receives $a$ from port 1 and $b$ from port 2, in any
order, then it has to go to state $t$.  A DFA would model this with
transitions: $s \xra{a} s_a \xra{b} t$ and $s \xra{b} s_b \xra{a}
t$. A DSA would contain two transitions $s \xra{ab, ba} t$ (and some
other transitions, if needed, to take care of $aa$, $bb$). To get a
more succinct notation we will use a $\parallel$ operator. We will
write $s \xra{a \parallel b} t$ to mean that at state $s$, when
both $a$ and $b$ are received, the automaton moves to $t$. When there
are several components, this notation leads to significant
succinctness, for instance $a_1 \parallel a_2 \cdots \parallel a_n$
stands for all the $n!$ permutations of $a_1$ to $a_n$. We will also
allow expressions of the form $a_1 a_2 \parallel b$, which stands for
the set of words $\{a_1 a_2 b, a_1 b a_2, b a_1 a_2\}$ which shuffles
$a_1a_2$ and $b$. We will now formalize this idea by enhancing DSA with
the $\parallel$ operator and then consider the problem of synthesizing
such extended DSA. We begin with some notation.%

% DFAs model
% a system consisting of $k$-components, where the $i^{th}$ component
% has sole control over actions of $\Sigma_i$. However every component
% can observe the actions generated by all other components. So, instead
% of modeling such a system as a network of automata, we consider it to
% be a single DFA over this multi-port alphabet. Therefore, for a start, 
% there is no ``independence'' between the ports: for example if
% action $a$ is controlled by process $1$ and action $b$ is controlled
% by process $2$, we cannot in general commute the sequence $ab$ with
% $ba$: it may be the case that process $2$ waits for process $1$ to
% emit an $a$. However, in the DFA, there could be
% specific parts where there is independence, in other words, there are
% states $s$, $t$ and a set of actions coming from different ports such
% that all interleavings among different ports are possible between $s$
% and $t$. We wish to extend the definition of a suffix-reading
% automaton with the aim to capture more succinctly these ``diamonds'' occurring in the
% DFA. %



\paragraph*{Notation} For a word $w \in \Sigma^*$, we write
$\proj{i}(w)$ for the projection of $w$ onto the set $\Sigma_i$. For
instance, if $\Sigma_1 = \{a_1, a_2\}, \Sigma_2 = \{b_1, b_2\}$ and
$w = a_1 b_1 a_2 a_1 b_2 b_2$, we have $\proj{1}(w) = a_1 a_2 a_1$ and
$\proj{2}{w} = b_1 b_2 b_2$.  We write $\partial w$ for the $k$-tuple
$(\proj{1}(w), \proj{2}(w), \dots, \proj{k}(w))$ of projections of $w$
onto each port $\Sigma_i$, and will call $\partial w$ the
\emph{decomposition} of $w$. Notice that if
$\partial w_1 = \partial w_2$ for two words $w_1, w_2$, then $w_1$ and
$w_2$ have the same order of events within each port, but could have a
different ordering between letters from different ports.%


\begin{definition} Let
 $\Sigma = \langle \Sigma_1, \Sigma_2, \dots, \Sigma_k \rangle$ be a
 multi-port alphabet. A \emph{multi-port (deterministic)
   suffix-reading automaton} (written \mdsa in short) $\Aa$ is a
 tuple $(Q, \Sigma, q^{init}, \delta, F)$ where $Q$, $q^{init}$ and
 $F$ are a finite set of states, the initial state and a set of
 accept states, respectively. The transition relation
 $\d \incl Q \times (\Sigma_1^* \times \Sigma_2^* \times \cdots
 \times \Sigma_k^*) \times Q$: each transition is of the form
 $(q, (u_1 \parallel u_2 \parallel \cdots \parallel u_k), q')$ where
 $u_i \in \Sigma_i^*$ (not all of them can be $\epsilon$).
 % We assume that for every two outgoing
 % transitions
 % $(q, (u_1, u_2, \dots, u_k), q_1)$ and
 % $(q, (v_1, v_2, \dots, v_k), q_2)$ there is at least one port $i$
 % such that $u_i$ and $v_i$ are suffix-incomparable, that is,
 % $u_i \not \sfx v_i$ and $v_i \not \sfx u_i$.
\end{definition}%

%\textcolor{red}{Jan 10, 2023. The first instinct is to say that the
% transition $q \xra{u_1 \parallel u_2 \cdots u_k} q'$ can be replaced
% with multiple transitions to get the usual DSA of the previous
% section. For instance $q \xra{a \parallel b} q'$ can be replaced
% with $q \xra{ab} q'$ and $q \xra{ba} q'$. But this is not correct in
% general!
% Suppose there is a third port. Consider a transition
% $q \xra{a \parallel b \parallel \epsilon} q'$. The third $\epsilon$
% is to say that we do not care what happens in that port. It does not mean
% that there is no input! We cannot just replace it with $\xra{ab}$
% and $\xra{ba}$. If we do so, the word $a c b$ at state $q$ would not
% move
% to $q'$ ($c$ is in third port), whereas it should. In fact, we
% cannot replace it with finitely many transitions: $a w b$ with $w$ a
% word using the third port alphabet is good for $a \parallel b
% \parallel \epsilon$. So, we need a completely fresh analysis for extending
% the parallel operation to DSA. We could keep this for future work.}%

\paragraph*{Semantics} $B=\langle B_1,B_2,\dots,B_{k} \rangle$ is a tuple of tapes, one for each port. Each tape has several tape heads, one for each unique transition label (corresponding to a `system requirement'), that track the previous `match' for each such requirement. 

\begin{enumerate}
\item All tapes are initially empty, with all the tape heads at zero. (call the head positions for the $B_i$ tape tracking transition label $j$, $H_{i}^{j}$)

\item When an input $a \in \Sigma_i$ is seen, the tape $B_i$ extends right by one position and store the value $a$.

\item A transition $(q, (u_1 \parallel u_2 \parallel \cdots \parallel u_k), q')$ with label $j$ (that is $u_1 \parallel u_2 \parallel \cdots \parallel u_k$) is \emph{matched} if every $B_i$ has $u_i$ as a suffix of its stored word, with either $H_{i}^{j}$ at the end of the tape or the entire $u_i$ present after $H_{i}^{j}$. Additionally, at least one $B_i$ has $H_{i}^{j}$ not at the end of the tape.

\item A matched transition is then \emph{triggered} and $\forall i, H_{i}^{j}$ moves to the end. 
\end{enumerate}

Note that the state of the mDSA can be recorded in an additional tape, which can read the current state and write its value. With an alphabet consisting of the the set of states $Q$, this tape (call it $B_S$) can effectively replicate the change of state on triggering a transitions

\begin{enumerate}
\item Initiate tape $B_S$ with value $q_i$

\item A transition (with label $j$) $\langle (q, (u_1, \dots, u_k), q') \rangle$ is matched if each $B_i (i\le k)$ has $u_i$ as suffix, the last letter of $B_{S}$ is $q$, and either $H_{i}^{j}$ is at the end or the entire $u_i$ is after $H_{i}^{j}$. Additionally, at least one tape is non-empty after $H_{i}^{j}$.

\item A matched transition is then triggered, moving each $H_{i}^{j}$-head of $B_i$ to the end. Additionally $B_{S}$ extends by one position to the right and stores $q'$.
\end{enumerate}


% Given states $q, q'$, we say that a word $w \in \Sigma^*$ matches a
% transition $(q, (u_1, \dots, u_k), q')$ at $q$ if
% $u_i \sfx \proj{i}(w)$ for all $i$. When the automaton is in state
% $q$ and reads a word $w$, we say that transition
% $\theta = (q, (u_1, \dots, u_k), q')$ is \emph{triggered} if $w$
% matches $\theta$ at $q$, and moreover, no proper prefix of $w$
% matches any outgoing transition at $q$.

\textcolor{red}{TODO: Lemma for equivalence between mDSA with states and mDSA with r/w tape}

 For a word $w \in \Sigma^*$, the run of \mdsa $\Aa$ on $w$ is given by:
 \begin{align*}
   q^{init} = q_0 \xra{w_1} q_1 \xra{w_2} \cdots \xra{w_{k}} q_k \xra{w_{k+1}}
 \end{align*}
 such that $w = w_1 w_2 \dots w_kw_{k+1}$ and there are transitions
 between $q_{i-1}$ and $q_i$ triggered on $w_i$ at $q_{i-1}$, for
 every $1 \le i \le k$. The run is \emph{accepting} if
 $w_{k+1} = \epsilon$ (no dangling letters) and $q_k$ is accepting. A
 word $w$ is accepted by \mdsa $\Aa$ if it has an accepting run. As
 usual, we define the language $\Ll(\Aa)$ to be the set of words that
 are accepted. Notice that we have extended the semantics of
 deterministic suffix-reading automata to the multi-port
 case. %Henceforth we will write mDSA for multi-port DSA.%

\begin{lemma}
 A multi-port DSA has a unique run over every word.
\end{lemma}%

\begin{theorem}
 For every DFA $M$ over a multi-port alphabet $\Sigma$, there exists
 a multi-port DSA $S_M$ such that $\Ll(M) = \Ll(S_M)$, and
 vice-versa.
\end{theorem}

\textcolor{red}{TODO: Fix proof. May need to synchronize tape heads}

\begin{proof} (Sketch.) For DFA to \mdsa: first complete the DFA, and
 then replace $q \xra{a} q'$ with
 $q \xra{(\e, \e, \dots, a, \dots, \e)} q'$ putting $a$ at the right
 port.%

 For \mdsa to DFA: replace $(u_1, u_2, \dots, u_k)$ with several
 transitions $q \xra{w} q'$ one for each $w$ such that
 $\partial w = (u_1, \dots, u_k)$. This gives a DSA. From previous
 result that DSA can be converted to DFA, we get DFA.%


\end{proof}%

\textcolor{red}{TODO: Complexity of membership for mDSA. Mostly linear with some pre-processing for string matching}



%%% Local Variables:
%%% mode: latex
%%% TeX-master: "mDSA"
%%% End:
